\documentclass[11pt]{book}
\usepackage[utf8]{inputenc}
\usepackage[T1]{fontenc}
\usepackage[english]{babel}
\usepackage{times}
\usepackage{tabularx}
\usepackage{colortbl}
\usepackage[pdfauthor={Environment and Climate Change Canada},pdfkeywords={GEM on a stick},pdftitle={GEM on a stick Manual - Version 5.1},colorlinks=true,linkcolor=blue,urlcolor=blue]{hyperref}
\usepackage[bottom=1.2in,top=1.4in]{geometry}

\textheight 8.6in
\topmargin -.3in
\textwidth 6.5in
%\oddsidemargin 0.2in  %( 8.5 - \textwidth )/2 - 1
\evensidemargin -0.1in %( 8.5 - \textwidth )/2 - 1
%\headheight 0pt
\headsep 1cm
\footskip 0.5in
\parskip 0.1in
\parindent 0em
\widowpenalty10000
\clubpenalty10000
\hyphenation{Canada}

\pretolerance=125 \tolerance=1000
\makeatletter
\makeatother

%%% package pour le pied de page
\usepackage{lastpage}
% /usr/share/texmf/doc/latex/fancyhdr/fancyhdr.dvi
\usepackage{fancyhdr}
\pagestyle{fancyplain}
\addtolength{\headheight}{2pt}
\lhead[\scriptsize\leftmark]{}
\rhead[]{\scriptsize\rightmark}
%\cfoot{\thepage}
%\pagestyle{fancy}
\fancyfoot[C]{\scriptsize  Environment and Climate Change Canada - GEM Manual - Page \thepage\ }
\fancyfoot[L,R]{}
%\fancyhead[L,C,R]{}
%\renewcommand{\headrulewidth}{0pt}

\begin{document}

%\frontmatter % The pages after this command and before the command
% \mainmatter, will be numbered with lowercase Roman numerals.

\thispagestyle{empty}

\title{Installing and running GEM \\ Version 5.1.a7}
\author{Environment and Climate Change Canada}
\date{\today}

\maketitle 


\thispagestyle{empty}\tableofcontents

%\mainmatter % This will restart the page counter and change the style to
             % Arabic numbers

\chapter{Installing GEM}

\section{Required tools and libraries}
\label{required-tools}

To compile and run GEM, you will need:

\begin{itemize}
\item Fortran and C compilers,
\item MPI/OpenMPI library (with development package),
\item BLAS library (with development package),
\item LAPACK library (with development package),
\item fftw3 library (with development package),
\item basic Unix utilities such as cmake (version 2.8.7 minimum), bash, sed, etc.
\end{itemize}

If those tools are not available on your computer, you can install them if
you have administrative rights. If not, check with your system administrator.

GEM was tested on:
\begin{itemize}
\item Fedora Core 29, with Intel 19.0.4.243 and cmake 3.14.4
\item Fedora Core 29, with gfortran 8.3.1 and cmake 3.14.4
\item Ubuntu 18.04, with gfortran 7.4.0 and cmake 3.10.2
\item Ubuntu 14.04, with gfortran 5.1.0 and cmake 2.8.12.2
\item Ubuntu 14.04, with Intel 16.0.1 and cmake 2.8.12.2
\end{itemize}

For gfortran/gcc, version 5 minimum is needed.

\section{First steps for the installation of GEM}
\label{compiling-gem}

GEM is ready to be compiled with Intel tools (proprietary compilers
available through a license) and GNU Fortran and C compilers (gcc and
gfortran: license under the GNU General Public License).

\begin{itemize} 
\item clone or download the git tar file of GEM at GitLab: \htmladdnormallink{http://gitlab.com/gem-ec/goas}{http\string://gitlab.com/gem-ec/goas},
\item a directory named \textit{goas}, is created,
\item \texttt{cd goas},
\item please note in that directory a file named \textit{README}, giving the
  same information as below.
\item execute the script named \textit{download-dbase.sh} or download
  directly the datafile at the following address and untar it: \\ \htmladdnormallink{http://collaboration.cmc.ec.gc.ca/science/outgoing/goas/gem\_dbase.tar.gz}{http\string://collaboration.cmc.ec.gc.ca/science/outgoing/goas/gem\_dbase.tar.gz}
\end{itemize}

Make sure the compilers and libraries are in the appropriate \textit{PATHS}:
you may have to initialize PATH and LD\_LIBRARY\_PATH.

\textbf{Examples for gcc/gfortran} (to be changed according to your setup):\\
- On Ubuntu:\\
export PATH=/usr/lib/openmpi/bin:\${PATH}\\
export LD\_LIBRARY\_PATH=/usr/lib/openmpi/lib:\${LD\_LIBRARY\_PATH}\\
- On Fedora:\\
export PATH=/usr/lib64/openmpi/bin:\${PATH}\\
export LD\_LIBRARY\_PATH=/usr/lib64/openmpi/lib\$:{LD\_LIBRARY\_PATH}\\

\textbf{Examples for PGI compilers} (to be changed according to your setup):\\
export PGI=/local/CUDA/PGI/pgi\\
export LM\_LICENSE\_FILE=\$PGI/license.dat\\
export PATH=\$PGI/bin:\$PGI/mpi/openmpi/bin:\${PATH}\\
export LD\_LIBRARY\_PATH=\$PGI/lib:\${LD\_LIBRARY\_PATH}\\

\textbf{Intel compilers}:\\
You may have to change the \texttt{MPI\_C\_COMPILER} and
\texttt{MPI\_Fortran\_COMPILER} variables according to your setup, for
example if you use OpenMPI instead of Intel MPI library.

\section{Compiling and installing GEM}
\label{building-gem}

GEM is configured by default to use gfortran and gcc compilers.

\begin{itemize}
\item \texttt{cd goas/sources}
\item \texttt{mkdir -p build}
\item \texttt{cd build}
\item \texttt{cmake ..}
\item if you want to compile with another compiler than gfortran, type:\\
  \texttt{cmake -DCOMPILER=intel ..} or\\
  \texttt{cmake -DCOMPILER=pgi ..}  or \\
  edit the \textit{sources/CMakeLists.txt} file to change the default
  compiler (line 9).
\item if you get errors messages (for example, compiler or MPI/OpenMPI not found), make
  sure these tools are exported in their respective \textit{PATHS} (see above)
\item if compiler or compile options and flags are not right:
\begin{itemize}
\item remove the content of the build directory: \texttt{rm -rf build/*}
\item make changes to the appropriate cmake file, such as : \\
  \textit{sources/Linux-x86\_64-gfortran.cmake} if you compile with gfortran or \\
  \textit{sources/Linux-x86\_64-intel.cmake} if you compile with Intel
\item \texttt{cmake ..} (or cmake -DCOMPILER=intel .. or cmake -DCOMPILER=pgi ..)
\end{itemize} 
\item \texttt{make -j}
\item \texttt{make install}
\item a directory named after the compiler you used, and the operating
  system you compiled on (\textit{work-OS\_NAME-COMPILER\_NAME}: for example
  \textit{work-Fedora-29\_x86-64-gfortran-8.3.1}) is created in the
  \textit{work} directory of GEM on a stick, and the following binaries are
  installed in a \textit{bin} directory in it: maingemdm, maingemgrid,
  mainyy2global, flipit, r.fstinfo, voir, editfst, fststat, cclargs\_lite,
  pgsm.
\end{itemize}

\chapter{Running GEM}

Running the GEM model simply consists of setting a few configuration files
then running a few scripts that will eventually take care of the three major
components of the model execution: 

\begin{enumerate} 
\item Prep: 	will prepare input date for the model
\item Model: 	will run the binary maingemdm (main time loop)
\item Output: 	will run post-processing of model output
\end{enumerate} 

\section{Running the default configuration of the model}
\label{running-gem}

In order to run the default configuration of the model:

\begin{itemize}
\item Move to the working directory created, named after the compiler you
  used and the operating system you compiled on, for example
  \textit{work-Fedora-29\_x86-64-gfortran-8.3.1}: \\ 
\texttt{cd goas/work/work-\textit{OS\_NAME-COMPILER\_NAME}}
\item \texttt{./runprep} (will create a directory named \textit{PREP})
\item \texttt{./runmod} (will run the model and create a directory named
  \textit{RUNMOD.dir} with output files)
\item or use \texttt{./runmod -ptopo 2x2x1} (to use 4 cpus for GEM-LAM,
  8 cpus for GEM-Yin-Yang - see below for details)

\item default setup uses the command \texttt{mpirun} to run the model. This
  command can be modified in the file \textit{scripts/Um\_model.sh}, line
  33, either in the original sources of the scripts
  (\textit{sources/orig-scripts/scripts}) or in the working directory
  (\textit{work/work-[OS\_NAME-COMPILER\_NAME]/scripts}). In the latter
  case, be advised that any changes to the scripts will be erased at the
  next \texttt{make install} step. So, if you want your changes to the
  scripts to be permanent, make them in the
  \textit{sources/orig-scripts/scripts} directory, and then rerun the
  command \texttt{make install} in the \textit{sources/build} directory.
 
\end{itemize}

\section{Tools to inspect the outputs}
\label{tools-outputs}

\begin{itemize} 
\item Use \textit{voir} to see what records are produced in the FST output files:

{\scriptsize\texttt{./voir -iment RUNMOD.dir/output/cfg\_0000/laststep\_0000000024/000-000/dm2009042700-000-000\_010}}

\item Use \textit{fststat} to look at statistical means on the records:

{\scriptsize\texttt{./fststat -fst RUNMOD.dir/output/cfg\_0000/laststep\_0000000024/000-000/dm2009042700-000-000\_010}}

%\item See details on these tools at section~\ref{utilities}.
\end{itemize}

\subsection{Tools to visualise the outputs}
\label{tools-SPI}

You can install SPI, a scientific and meteorological virtual globe offering
processing, analysis and visualisation capabilities, with a user interface
similar to Google Earth and NASA World Wind, developed by Environment
Canada.  

SPI can be downloaded at
\htmladdnormallink{http://eer.cmc.ec.gc.ca/software/SPI}{http\string://eer.cmc.ec.gc.ca/software/SPI}.\\
Follow instructions found in \textit{INSTALL.txt}, in the 7.12.0 directory.

You can also install xrec, a visualisation program used to display 2D
meteorological fields stored in the RPN standard file format, developed by
Research Informatics Services, Meteorological Research Division, Environment
and Climate Change Canada. xrec on a stick can be cloned or downloaded at:
\htmladdnormallink{https://gitlab.com/gem-ec/xoas}{https\string://gitlab.com/gem-ec/xoas}.

\section{Configuration Files}
\label{config-files}

The execution of all three components of GEM is highly configurable through
the use of three configuration files called:

\begin{enumerate} 
\item gem\_settings.nml: file containing some namelists to configure the model execution
\item outcfg.out: 	file use to configure the model output
\item configexp.cfg: 	file use to configure the execution shell environment
\end{enumerate} 

Examples of these files can be found in the test cases in the
\textit{work/configurations} directory.

%TO BE MODIFIED
% A complete documentation of these files can be found in \htmladdnormallink{GEM web site}{http\string://iweb.cmc.ec.gc.ca/~armncpi/gem/index-en.php}.

A fourth configuration file, named physics\_input\_table, is used for GEM\_cfgs,
GY\_cfgs and LAM\_cfgs test cases.

\section{Running different configurations}

The configuration used by default when running the model is in the directory
\textit{work/configurations/GEM\_cfgs}, but you will find other configurations in the
\textit{work/configurations} directory:

\begin{itemize}
\item GY\_cfgs: using GEM Yin-Yang
\item LAM\_cfgs: using GEM LAM
\item Bubble\_cfgs\_bubble: theoretical case %: \textit{décrire}.
\item Schaer\_cfgs: Schaer mountain wave
\item GEM\_theo\_cfgs: theoretical test cases% (see details in \htmladdnormallink{GEM web site}{http\string://iweb.cmc.ec.gc.ca/~armncpi/gem/index-en.php}),
\end{itemize} 

You can use them by giving as an option the directory where the
configurations files are situated:

To prepare the input: \texttt{./runprep -dircfg cfg\_dir}\\
(for example:  \texttt{./runprep -dircfg GY\_cfgs})

To run the model: \texttt{./runmod -dircfg cfg\_dir}
(for example:  \texttt{./runmod -dircfg GY\_cfgs})

To run the model on many cpus \texttt{./runmod -dircfg cfg\_dir  -ptopo 2x2x1} (to use 4 cpus for GEM-LAM,
  8 cpus for GEM-Yin-Yang - see below for details)

\section{Running your own configuration}

Put the three configurations files (gem\_settings.nml, outcfg.out and
configexp.cfg) in a directory structure such as: \textit{exp\_dir/cfg\_0000}
in the \textit{work/configurations} directory.

The master directory name (\textit{exp\_dir} in the example above) can be
any valid directory name. However, the second directory must have the name
cfg\_0000.

Then run the two scripts with the following commands:

To prepare the input: \texttt{./runprep -dircfg exp\_dir}

To run the model: \texttt{./runmod -dircfg exp\_dir}

To run the model on many cpus \texttt{./runmod -dircfg exp\_dir  -ptopo 2x2x1} (to use 4 cpus for GEM-LAM,
  8 cpus for GEM-Yin-Yang - see below for details)

\section{Modifying the grid and getting meteorological data}

You can use the script named \textit{grille.sh}, in the \textit{scripts}
directory, to define your own grid and visualise it with SPI (see
section~\ref{tools-SPI} above how to get it).

For this, copy one of the \textit{gem\_settings.nml} files located in the
different configurations directories, edit it, and then run the command
\texttt{grille.sh -spi}.

If you want geophysical fields and historical meteorological data for the
region you defined in that new grid, contact us.

\chapter{Other Utilities}
\label{utilities}

\section{pgsm}

pgsm is a utility designed to perform horizontal interpolations and basic
arithmetic operations on RPN standard files.

Input files must be RPN standard files. Output files may be RPN standard
files (random or sequential), binary FORTRAN sequential files, or formatted
ASCII files.

PGSM can:
\begin{itemize}
\item Interpolate data on various geographical projections, compressed or not.
\item Interpolate wind components UU and VV with respect to the scale and orientation of the output grid.
\item Perform symmetric or antisymmetric extrapolations from an hemispheric grid.
\item Compute thicknesses between 2 levels.
\item Compute precipitation amounts between 2 forecast hours.
\item Extract low, mid and high clouds.
\item Perform mathematical operations on one field or more.
\item Compute latitudes and longitudes from the X-Y coordinates of a grid.
\item Read navigational records of latitudes-longitudes (grid type Y) or
  grid coordinates (grid type Z) in an input or output file and interpolate
  values from these coordinates.
\end{itemize}

Example:\\
\texttt{./pgsm -iment <input FST> -ozsrt <output FST> -i <pgsm.directives>}

\section{editfst}

editfst is a utility used for editing and copying records from RPN standard
files into a new or an existing RPN standard file. It can do a straight
(complete) copy of the input file or it can copy records selectively as
indicated from the standard input or from a file of directives named in the
-i option.

Example:\\
\texttt{./editfst -s <input FST> -d <output FST> -i <editfst.directives>}

More details on these utilities will be available soon.

\end{document}
